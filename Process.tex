\documentclass[10pt]{article}
\usepackage{geometry}
\geometry{a4paper}
\usepackage{hyperref}


\title{Cook-E Process Description}
\author{Sam Crow (\texttt{samcrow}), [Name to be found] (\texttt{wszzj}), Shan Yang (\texttt{shany3}),\\ Xianzhe Peng (\texttt{pxz2012}), Tyler Jacoby (\texttt{tsjacoby}),\\ Kyle Woo (\texttt{kylew5}), Carson Lipscomb (\texttt{carsonal})}
\begin{document}
\maketitle{}


\section{Software toolset}

\subsection{Mobile application}

The mobile application will be written in Java and will run on Android. It will support Android versions 4.0 and higher
(API levels 14 and higher). According to Google's statistics, this will make the application compatible with 96% of
Android devices that have recently provided their information to Google.

The development team chose to target Android because it is widely used and some team members have experience developing
Android applications. Java is the primary language used for Android development, and all team members have experience with
it.

\subsection{Version control}

The project will use Git for version control, with central repositories hosted using on github.com.

The project will use the issue tracking and wiki features of the GitHub repository for bug tracking and development documentation.
\subsection{Testing}

All software components should be accompanied by unit tests developed in parallel.
Each team member should write tests for their code.
The testing coordinator (see below) will develop integration tests and user interface tests for the application.

\section{Group Dynamics}

\subsection{Core Roles}

Some team members have core roles that give them particular responsibilities in the project.
All team members will still work on developing the application.

\paragraph{Project Manager: Xianzhe Peng}

The responsibilities of the project manager include:

\begin{itemize}
\item Leading team meetings
\item Resolving disputes (see Dispute Resolution below)
\item Updating the System Requirements Specification and System Design Specification and Plan
for later deliverables
\end{itemize}

\paragraph{Testing Coordinator: Zijin Zhang}

The responsibilities of the testing coordinator include:

\begin{itemize}
\item Helping other team members set up unit testing for their code
\item Writing integration and user interface tests
\item Handling regressions, either by making fixes for small problems or opening
issues and alerting the responsible team members for larger problems
\end{itemize}

\paragraph{Infrastructure Coordinator: Sam Crow}

The responsibilities of the infrastructure coordinator include:

\begin{itemize}
\item Maintaining the repository and continuous integration system
\item Helping other team members use the version control system
\item Checking pull requests
\begin{itemize}
    \item Confirming that new code passes tests and follows the code style guidelines
    \item Merging code from pull requests
\end{itemize}
\item Maintaining development documentation
\end{itemize}

\paragraph{User Documentation Coordinator: Carson Lipscomb}

The responsibilities of the user documentation coordinator include:

\begin{itemize}
\item Defining the structure of user documentation
\item Ensuring that user documentation is consistent and of sufficient quality
\item Before each release, checking that all user documentation is accurate
\end{itemize}

\subsection{Dispute Resolution}

\paragraph{Features}

After the core features in the software requirements are finalized, no feature
may be added to the project unless five of the seven team members vote in favor
of adding it.

At any time, the project manager may decide to remove a feature to ensure that
the project is completed on time.

\paragraph{Other Disputes}

If a dispute cannot be resolved through deliberation, the project manager should
make an executive decision to resolve the dispute. The decision may be overridden
if five or more of the seven team members vote to override it.

\subsection{Justification}

This system of roles ensures that important tasks are performed but does not
restrict the development tasks that team members can work on. The feature dispute
resolution process is designed to reduce the risk of feature creep. The resolution
process for other disputes is designed to be fast while still allowing the team
to override project manager decisions in unusual circumstances.

\section{Timeline}

\begin{itemize}
\item 2016-01-22: Software requirements specification due
\item 9 days: Team splits into user interface design, algorithms, and application groups
\begin{itemize}
    \item 3 days: user interface design group develops detailed user interface design,
    other groups design data and algorithms
    \item 6 days: Application group starts implementing user interface, other groups
    finish data and algorithm design
\end{itemize}
\item 2016-02-01: Software design specification due
\item 6 days: Prepare zero-feature release
\begin{itemize}
    \item 2016-02-02: Infrastructure coordinator creates project
    \item Application group implements user interface
    \item Other team members, led by the infrastructure coordinator and the user
    documentation coordinator, write initial documentation
\end{itemize}
\item 2016-02-08: Zero-feature release due, algorithm design must be complete
\item 11 days: Implement features
\item 2016-02-19: Beta release due
\item 6 days: Prepare for feature-complete release
\begin{itemize}
    \item 2016-02-20: Testing coordinator identifies deficiencies in tests and asks
    the responsible team members to improve them, user documentation coordinator
    identifies user documentation deficiencies and arranges for them to be fixed
    \item 2016-02-25: Testing coordinator runs coverage tool and records results
\end{itemize}
\item 2016-02-26: Feature-complete release due
\end{itemize}

\section{Risk Summary}

\subsection{Scheduling Algorithm}

The scheduling algorithm is the most critical part of this project. It needs to
solve a problem with many constraints. The most difficult part of the algorithm is being able to adjust to users' varying times at completing different steps. If users do not perceive that the
algorithm is better at scheduling than a person, they will not use the application.

To improve the probability of the team developing a good algorithm, we have allocated a long period of time
early in the project for algorithm design. Part of the team will work primarily
on the algorithm.

\subsection{Acquiring Recipe Data}

A key part of this project is acquiring enough recipes to be useful for a user who may want to cook any combination of dishes. There may be difficulty in finding data sources for the recipes and then creating an automated way to pull that data and putting it into our own database (or streaming it from the source).

For the early prototypes we will manually add around 10 recipes and then spend time adding recipes to the database after the scheduling algorithm is finished

\subsection{Testing Difficulties}

Testing the whole application and its user interface will be more complex than
small-scale unit testing. The testing coordinator will be responsible for
setting up and maintaining user interface tests. If automated user interface
testing is not feasible, the testing coordinator will develop procedures and
test the user interface manually.

\end{document}
